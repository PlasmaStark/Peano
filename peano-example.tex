\documentclass{beamer}

\usetheme{peano}

\title[Modern Cryptography]{Modern Cryptography}
\subtitle{From One-Time Pad to Post-Quantum}
\author[N. Surname]{Richard A. Euler}
\institute{Politecnico di Torino}
\date[EC26]{Eurocrypt 2026}

\begin{document}

\begin{frame}
  \titlepage
\end{frame}

% ===== SECTION 1 =====
\section{Introduction to Cryptography}

\begin{frame}{What is Cryptography?}
\begin{block}{Definition}
Cryptography is the study of techniques for securing communication in the presence of adversaries.
\end{block}

\begin{itemize}
\item<1-> Confidentiality
\item<2-> Integrity
\item<3-> Authenticity
\item<4-> Non-repudiation
\end{itemize}
\end{frame}

\begin{frame}{Types of Cryptography}
\begin{exampleblock}{Symmetric}
Same key for encryption and decryption (e.g. AES).
\end{exampleblock}

\begin{alertblock}{Asymmetric}
Public and private key pair (e.g. RSA, ECC).
\end{alertblock}
\end{frame}

% ===== SECTION 2 =====
\section{Classical Results}

\begin{frame}
\frametitle{There Is No Largest Prime Number}
\framesubtitle{Proved with \textit{reductio ad absurdum}.} 

\begin{theorem}
There is no largest prime number.
\end{theorem}

\begin{enumerate}
  \item<1-| alert@1> Suppose $p$ is the largest prime number.
  \item<2-> Let $q$ be the product of the first $p$ numbers.
  \item<3-> Then $q+1$ is not divisible by any of them.
  \item<4-> But $q + 1$ is greater than $1$, so it must have a prime factor not among the first $p$.
\end{enumerate}
\end{frame}

\begin{frame}{Euler’s Theorem}
We know that:\footnote{Stating the theorem is an exercise left to the reaer.}
\[
a^{\phi(n)} \equiv 1 \pmod{n} \quad \text{if } \gcd(a,n)=1
\]

\pause

Useful in RSA encryption:
\[
c = m^e \mod n \quad\text{and}\quad m = c^d \mod n
\]
\end{frame}

% ===== SECTION 3 =====
\section{Modern Schemes}

\begin{frame}{Elliptic Curve Cryptography}
\begin{itemize}
\item Based on algebraic structure of elliptic curves over finite fields
\item Smaller key sizes than RSA for same security level
\item Standard in many protocols: TLS, Signal, Bitcoin
\end{itemize}
\end{frame}

\begin{frame}{Post-Quantum Cryptography}
\begin{alertblock}{Motivation}
Quantum computers threaten RSA, ECC, and DH.
\end{alertblock}

\begin{exampleblock}{Lattice-Based Cryptography}
Resistant to Shor's algorithm. Used in NIST PQC finalists.
\end{exampleblock}
\end{frame}

% ===== SECTION 4 =====
\section{Conclusion}

\begin{frame}{Summary}
\begin{itemize}
\item Cryptography secures digital communication
\item From classical to quantum-resistant schemes
\item Future: Hybrid and post-quantum protocols
\end{itemize}
\end{frame}

\begin{frame}{References}
\begin{itemize}
\item D. Boneh, V. Shoup, A Graduate Course in Applied Cryptography
\item NIST PQC Project — \url{https://csrc.nist.gov/Projects/post-quantum-cryptography}
\item Bruce Schneier, Applied Cryptography
\end{itemize}
\end{frame}

\begin{frame}{Thank you!}
\begin{center}
Questions?

\vspace{2em}
\end{center}
\end{frame}

\end{document}